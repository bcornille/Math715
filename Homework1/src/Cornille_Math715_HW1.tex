\documentclass{article}

\usepackage{myhw}
\usepackage{mymacros}

\newcommand\NoIndent[1]{%
  \begingroup
  \par
  \parshape0
  #1\par
  \endgroup
}

\renewcommand\thesection{}
\renewcommand\thesubsection{}

\lhead{Math 715}
\chead{Homework 1}

\begin{document}

\begin{enumerate}
\NoIndent{\section{Theoretical Background}}

	\item $a$ is an unknown parameter.
		On $\mathbb{R}^2$, find out the region where the following equation is
		elliptic, hyperbolic, or parabolic, and study their dependence on $a$.
		\begin{equation*}
			(a+x)\hopdf{2}{u}{x} + 2xy\frac{\partial^2 u}{\partial x \partial y}
				- y^2\hopdf{2}{u}{y} = 0
		\end{equation*}

\NoIndent{
	To classify a second order PDE of the form
	\begin{equation*}
		a(x,y)\hopdf{2}{u}{x} + 2b(x,y)\frac{\partial^2 u}{\partial x \partial y} + c(x,y)\hopdf{2}{u}{y}
	\end{equation*}
	we must consider $b^2 - ac$.
	For the given equation this yeilds $x^2 y^2 + (a+x)y^2$.
	Thus the regions depend on the parameter $a$.
	When $a > \frac{1}{4}$ the equation is hyperbolic everywhere.
	When $a = \frac{1}{4}$ the equations is parabolic at $x = -\frac{1}{2}$ and hyperbolic elsewhere.
	When $a < \frac{1}{4}$ then
	\begin{equation*}
		\begin{cases}
			x = \displaystyle\frac{-1 \pm \sqrt{1-4a}}{2}  & \mathrm{parabolic} \\
			\displaystyle\frac{-1 - \sqrt{1-4a}}{2} < x < \displaystyle\frac{-1 + \sqrt{1-4a}}{2}
				& \mathrm{elliptic} \\
			otherwise & \mathrm{hyperbolic}
		\end{cases}
	\end{equation*}
}

	\item We prove the maximum principle for the solution to the Poisson equation
		\begin{equation*}
			\begin{cases}
				-\Delta u = f & \mathrm{in\ } \Omega \\
				u = g & \mathrm{on\ } \partial\Omega
			\end{cases}
		\end{equation*}
		Prove that there exists a constant $C$ depending only on $\Omega$ such that
		\begin{equation*}
			\max_{\bar{\Omega}}|u| \leq C\left(\max_{\partial\Omega}|g| + \max_{\bar{\Omega}}|f|\right)
		\end{equation*}
		Here $\Omega$ is a bounded open subset of $\mathbb{R}^n$ with $\Gamma$ being its boundary.
		$\bar{\Omega} = \Omega \cup \partial\Omega$ is its closure.
		Several hints:
		\begin{enumerate}
			\item You could use the following property:
				assume $v$ is subharmonic, i.e.\ $-\Delta v \leq 0$ in $\Omega$, then
				\begin{equation*}
					v(x) \leq \frac{1}{V}\int_{B(x,r)}v(y)\d y
				\end{equation*}
				Here $B(x,y)$ is a ball centered at $x$ with radius r, and $V$ is the volume of the ball.
				Given this, you could prove the maximum of $v$ is achieved at the boundary $\partial\Omega$.
			\item Try to show the function $u + \displaystyle\frac{|x|^2}{2n}\lambda$ is subharmonic.
			Here $\lambda = \displaystyle\max_{\bar{\Omega}}|f|$.
		\end{enumerate}

\NoIndent{
	
}

\NoIndent{\section{Finite Differencing}}

	\item \begin{enumerate}
		\item Prove
			$\Delta_- + \Delta_+ =
				\left(\mathcal{E}^{-\frac{1}{2}} + \mathcal{E}^\frac{1}{2}\right)\Delta_0$
			Here $\mathcal{E}$ is the shifting operator, $\left(\mathcal{E}u\right)_j = u_{j+1}$,
			and the definitions for $\Delta_+$ and $\Delta_-$ are consistent with what we has in class.
			$\left(\Delta_0 u\right)_j = u_{j+\frac{1}{2}} - u_{j-\frac{1}{2}}$
		\item Determine the constants $c$ and $d$ so that
			\begin{align*}
				\partial_x^2 u(x) - \frac{1}{h^2}\left(\Delta_+^2 - \Delta_+^3\right)u(x)
					&= ch^2\partial_x^4 u(x) + \bigO(h^3) \\
				\partial_x^2 u(x) - \frac{1}{h^2}\Delta_0^2 u(x) &= dh^2\partial_x^4 u(x) + \bigO(h^4)
			\end{align*}
			Here we assume $u(x)$ is smooth enough.
		\item The two identities above tell you how to approximate $\partial_x^2$
			using forward differencing and central differencing.
			How many grid points do you need to get the second order approximation
			respectively using these two methods?
	\end{enumerate}

	\item Write a computer code (using your favorite language) to determine, to highest possible order,
		a finite difference approximation to $u''(x)$ based on the 5-point stencil
		$\{x-h,x-\frac{1}{2}h,x,x+h,x+2h\}$
		\begin{equation*}
			u''(x) \approx
				c_0 u(x-h) + c_1 u\left(x-\frac{h}{2}\right) + c_2 u(x) + c_3 u(x+h) + c_4 u(x+2h)
		\end{equation*}
		\begin{enumerate}
			\item Compute $c_j$.
			\item Check the order.
		\end{enumerate}

\NoIndent{Finite difference method for an elliptic equation}

	\item We used Fourier method for stability analysis in 1D in class.
		Carry out the same analysis for 2D.

	\item Derive the expldit formulae for Green's functions in 1D
		and prove they are piecewise linear function.

	\item In 2D, to compute the Poisson equation, $u'' = f$
		with zero boundary condition on a rectangular domain,
		we discretize the domain by even grid points with mesh size $h$.
		Denote $A$ the associated discretization matrix with central differencing (5-stencil).
		Show $||A^{-1}||_\infty$ is bounded independent of $h$.
		Explain why it suggests that the order of accuracy of the numerical method is second order.
		(Hint:
			(a) $\left\|A^{-1}\right\|_\infty =
				\displaystyle\sup\frac{\left\|A^{-1}v\right\|_\infty}{\left\|v\right\|_\infty}$.
			(b) Fundeamental theorem for numerical convergence.)

	\item Prove the discrete version of the Poincar\'e inequality,
		\begin{equation*}
			\sum_{m,n}\left|U_{m,n}\right|^2 \leq \sum_{m,n}\left|\partial_x U_{m,n}\right|^2
		\end{equation*}
		Here $U$ is a matrix on 2D with zero boundary condition,
		and $\partial_x U_{m,n}$ is the forward Euler representation of differentiation,
		definde by, $\partial_x U_{m,n} = \frac{1}{h}\left(U_{m+1,n} - U_{m,n}\right)$.
\end{enumerate}

\end{document}
