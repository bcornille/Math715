\documentclass{article}

\usepackage{myhw}
\usepackage{mymacros}

\newcommand\NoIndent[1]{%
  \begingroup
  \par
  \parshape0
  #1\par
  \endgroup
}

\renewcommand\thesection{}
\renewcommand\thesubsection{}

\DeclareMathOperator{\Span}{span}

\lhead{Math 715}
\chead{Homework 2}

\begin{document}

\begin{enumerate}
\NoIndent{\section{Variational Formulation}}

	\item Suppose $u \in V$ is the solution to the following problem in its variational form
		\begin{equation*}
			b(u,v) = l(v) \qquad v \in V
		\end{equation*}
		where $b(u,v)$ is a bilindear form that maps $V \times V$ to $\mathbb{R}$.
		$l$ is a linear operator that maps $V$ to $\mathbb{R}$.
		Here $V$ is a Hilbert space.
		To obtain a numerical solution, we use the Galerkin method, and search for a solution in a subspace
		$V_h = \Span\left\{\phi_1,\phi_2,\dots,\phi_N\right\} \subset V$
		\begin{equation*}
			b(u_h,v) = l(v) \qquad v \in V_h
		\end{equation*}
		In the linear algebra form, it is written as
		\begin{equation*}
			\mat{A}\ccdot\arr{U} = \arr{F}
		\end{equation*}
		\begin{enumerate}
			\item Find the specific form of $\mat{A}$ and $\arr{F}$.
\NoIndent{
	\begin{equation*}
		\left(\mat{A}\right)_{i,j} = b\left(\phi_j,\phi_i\right)
	\end{equation*}
	and
	\begin{equation*}
		\left(\arr{F}\right)_i = l\left(\phi_i\right)
	\end{equation*}
}
			\item If $b$ is symmetric, show $\mat{A}$ is symmetric.
\NoIndent{
	Assume $b$ is symmetric, i.e.\ $b\left(\phi_j,\phi_i\right) = b\left(\phi_i,\phi_j\right)$.
	Then $\left(\mat{A}\right)_{i,j} = \left(\mat{A}\right)_{j,i}$, i.e.\ $\mat{A} = \mat{A}^T$,
	so $\mat{A}$ is symmetric.
}
			\item Give the definnition of the coercivity condition,
				and show if $b$ is coercive, $\mat{A}$ is positive definite.
\NoIndent{
	Assume $b$ is coercive, i.e.\ $\gamma\|u\|^2 \leq b(u,u)$.
	We can transform the basis,
	$\phi_1,\phi_2,\dots,\phi_N \rightarrow \tilde{\phi}_1,\tilde{\phi}_2,\dots,\tilde{\phi_N}$,
	such that $b\left(\tilde{\phi}_j,\tilde{\phi}_i\right) = 0$ for $i \neq j$,
	which is equivalent to diagonalizing $\mat{A}$.
	(We know that $\mat{A}$ is diagonalizable since our problem is well-posed,
	thus $\mat{A}$ is invertible.)
	Now with this diagonalized basis set, the eigenvalues of $\mat{A}$ are
	$\lambda_i = b\left(\tilde{\phi}_i,\tilde{\phi}_i\right)
		\geq \gamma\left\|\tilde{\phi}_i\right\|^2 > 0$.
	Therefore, $\mat{A}$ is positive definite.
}
			\item Still assume $b$ is coercive and denote $M$ and $\gamma$ the bounded coefficient and
				coercive coefficient, respectively, show
				\begin{equation*}
					\left\|u - u_h\right\| \leq \frac{M}{\gamma}\inf_{v\in V_h}\|u - v\|
				\end{equation*}
\NoIndent{
	Assume $b$ is coercive and bounded, i.e.\ $\gamma\|u\|^2 \leq b(u,u)$ and $b(u,v) \leq M \|u\|\|v\|$.
	Using $b\left(u-u_h,v\right) = 0$ $\forall v \in V_h$, it can be shown that
	$b\left(u-u_h,u-v\right) = b\left(u-u_h,u-u_h\right)$.
	Then,
	\begin{align*}
		\gamma\left\|u-u_h\right\|^2 \leq b\left(u-u_h,u-u_h\right) &= b\left(u-u_h,u-v\right)
			\leq M \left\|u-u_h\right\|\|u-v\| \qquad \forall v \in V_h \\
		\left\|u - u_h\right\| &\leq \frac{M}{\gamma}\inf_{v\in V_h}\|u - v\|
	\end{align*}
}
			\item For the following equation in $\Omega \in \mathbb{R}^2$
				with zero Dirichlet boundary condition ($\Omega$ is compactly supported),
				write down its variational form, determine the space $V$,
				and show whether $b$ satisfies the coercivity condition.
				Find $M$ and $\gamma$, respectively.
				\begin{equation*}
					-\divg(a\grad{u}) + cu = f
				\end{equation*}
				with $0 < \ushort{a} \leq a < \bar{a}$ and $0 \leq \ushort{c} \leq c < \bar{c}$.
\NoIndent{
	Find some $u \in H_0^1$ such that
	\begin{equation*}
		\left<a\grad{u},\grad{v}\right> + \left<cu,v\right> = \left<f,v\right> \qquad \forall v \in H_0^1
	\end{equation*}
	i.e.
	\begin{equation*}
		b(u,v) = \left<a\grad{u},\grad{v}\right> + \left<cu,v\right>
	\end{equation*}
	To check coercivity,
	\begin{equation*}
		b(u,u) = \left<a\grad{u},\grad{u}\right> + \left<cu,u\right> \geq
			\ushort{a}\left<\grad{u},\grad{u}\right> + \ushort{c}\left<u,u\right> =
			\ushort{a}\|u\|_{H^1}^2 + \left(\ushort{c}-\ushort{a}\right)\|u\|_{L^2}^2 \geq
			\left(\ushort{a}+\ushort{c}\right)\|u\|_{H^1}^2
	\end{equation*}
	so $\gamma = \ushort{a}+\ushort{c}$.
	For the bounded condition,
	\begin{equation*}
		b(u,v) = \left<a\grad{u},\grad{v}\right> + \left<cu,v\right> \leq
			\bar{a}\left<\grad{u},\grad{v}\right> + \bar{c}\left<u,v\right>
	\end{equation*}
	noting that $\left<\grad{u},\grad{v}\right> \leq \|u\|_{H^1}\|v\|_{H^1}$
	and $\left<u,v\right> \leq \|u\|_{H^1}\|v\|_{H^1}$,
	\begin{equation*}
		b(u,v) \leq \left(\bar{a}+\bar{c}\right)\|u\|_{H^1}\|v\|_{H^1}
	\end{equation*}
	so $M = \bar{a}+\bar{c}$.
}
			\item For the problem above, if we choose $V_h$ to be a piecewise linear function space, show
				\begin{equation*}
					\left\|u - u_h\right\|_{H^1} = \bigO(h)
				\end{equation*}
				and that if $a$ is a constant and $c = 0$
				\begin{equation*}
					\left\|u - u_h\right\|_{L_2} = \bigO(h^2)
				\end{equation*}
\NoIndent{
	Using $b$ is coercive and bounded along with previous analysis, it can be shown that
	\begin{equation*}
		\left\|u - u_h\right\|_{H^1} \leq \frac{M}{\gamma}\|u - v\|_{H^1} \qquad \forall v \in V_h
	\end{equation*}
	Simply pick $v = Iu$, where $Iu$ is the interpolation of $u$
	using piecewise linear functions ($Iu \in V-h$).
	Then
	\begin{equation*}
		\left\|u - u_h\right\|_{H^1} \leq \frac{M}{\gamma}\|u - Iu\|_{H^1} = \bigO(h)
	\end{equation*}
}
			\item For the same equation in $\Omega \in \mathbb{R}^2$ with Neumann boundary contition
				\begin{equation*}
					\left.\partial_n u\right|_{\partial\Omega} = g
				\end{equation*}
				Write down its variational form, determine $V$,
				and show whether $b$ satisfies the coercivity condition.
		\end{enumerate}

\NoIndent{\section{Euler-Bernoulli equation}}

	\item Consider the Euler-Bernoulli equation
		\begin{equation*}
			\hopdf{4}{u}{x} = f(x) \qquad 0 < x < 1
		\end{equation*}
		It is used to describe the deflection of $u$ of a clamped beam
		subject to a transversal force with intensity $f$.
		\begin{enumerate}
			\item Show the equivalent varational form would be to find $u$ such that
				\begin{equation*}
					\left<u'',v''\right> = \left<f,v\right> \qquad \forall v \in V
				\end{equation*}
				where $V = \left\{v:v\in C_1[0,1],v(0)=v'(0)=v(1)=v'(1)=0,
					v\mathrm{\ piecewise\ continuous\ and\ bounded}\right\}$
			\item For an interval, $I = [a,b]$, define $P_3(I) =
				\left\{v:v(x) = c_0 + c_1 x + c_2 x^2 + c_3 x^3, x \in I\right\}$.
				Show that $v \in P_3(I)$ is uniquely determined by the values
				$v(a)$, $v'(a)$, $v(b)$, and $v'(b)$.
				Find the corresponding local basis functions.
				(Hint: count the nuber of degrees of freedom and use the values to fix the coefficients.)
			\item Construct a finite-dimensional subspace $V_h$ consisting of
				piecewise cubic polynomials on the mesh $0 = x_0 < x_1 < \cdots < x_{N+1} = 1$.
			\item Derive the error estimate
				\begin{equation*}
					\left\|\left(u - u_h\right)''\right\|_2 \leq \left\|\left(u - v\right)''\right\|_2
						\qquad \forall v \in V_h
				\end{equation*}
				You are given the estimate that cubic Hermite interpolation of $u$,
				denoted as $I_h u \in V_h$, satisfies the following
				\begin{equation*}
					\left\|u''(x) - \left(I_h u\right)''(x)\right\| \lesssim
						h^2\max_{0\leq\xi\leq 1}\left|u^{(4)}(\xi)\right|
				\end{equation*}
				show that
				\begin{equation*}
					\left\|\left(u - u_h\right)''\right\| \leq
						Ch^2\max_{0\leq\xi\leq 1}\left|u^{(4)}(\xi)\right|
				\end{equation*}
			\item Write a computer program to solve
				\begin{equation*}
					\begin{cases}
						\displaystyle\hoddf{4}{u}{x} = g(x) \\
						u(0) = u'(0) = u'(1) = u'(1) = 0
					\end{cases}
				\end{equation*}
				If we use
				\begin{equation*}
					g(x) = \hoddf{4}{}{x}\left(e^x x^2(1 - x)^2\right) =
						e^x\left(x^4 + 14x^3 + 49x^2 + 32x - 12\right)
				\end{equation*}
				the exact solution is $u(x) = e^x x^2(1 - x)^2$.
				\begin{enumerate}
					\item Give a brief description of your algorithm, in particular,
						the method you use to evaluate the load vector $\arr{b}$
						(choose your favorite numerical integral method,
						but make sure the error here is not too big,
						and the error from $\mat{A}$ still dominates)
					\item Tabulate the max-norm errors $e_N = \max\left|u_h(x_j) - u(x_j)\right|$
						and show the numerical convergence order by performing linear regression of
						$\log{e_N} \mathrm{\ vs\ } \log{N}$
					\item Plot your finite element solution $u_h$ along with the real solution.
					\item Attach your code.
				\end{enumerate}
		\end{enumerate}
\end{enumerate}

\end{document}
